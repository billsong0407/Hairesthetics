\documentclass{article}
\usepackage{pdflscape}
\usepackage{booktabs}
\usepackage{xcolor}
\usepackage{tabularx}
\usepackage{hyperref}
\usepackage[a4paper, total={6in, 8in}]{geometry}
\usepackage{changepage}
\usepackage[normalem]{ulem}

\newcounter{acrreqnum} % Access Requirement Number
\newcommand{\rtheacrreqnum}{P\theacrreqnum}
\newcounter{irreqnum} % Integrity Requirement Number
\newcommand{\rtheirreqnum}{P\theirreqnum}
\newcounter{prrreqnum} % Privacy Requirement Number
\newcommand{\rtheprrreqnum}{P\theprrreqnum}
\newcounter{aurreqnum} % Audit Requirement Number
\newcommand{\rtheaurreqnum}{P\theaurreqnum}

\hypersetup{
    colorlinks=true,       % false: boxed links; true: colored links
    linkcolor=red,          % color of internal links (change box color with linkbordercolor)
    citecolor=green,        % color of links to bibliography
    filecolor=magenta,      % color of file links
    urlcolor=cyan           % color of external links
}

\title{Hazard Analysis\\Hairesthetics}

\author{Team 18 \\ Charlotte Cheng
        \\ Marlon Liu
        \\ Senni Tan
        \\ Qiushi Xu
        \\ Hongwei Niu
        \\ Bill Song
}

\date{\today}

%% Comments

\usepackage{color}

\newif\ifcomments\commentstrue %displays comments
%\newif\ifcomments\commentsfalse %so that comments do not display

\ifcomments
\newcommand{\authornote}[3]{\textcolor{#1}{[#3 ---#2]}}
\newcommand{\todo}[1]{\textcolor{red}{[TODO: #1]}}
\else
\newcommand{\authornote}[3]{}
\newcommand{\todo}[1]{}
\fi

\newcommand{\wss}[1]{\authornote{blue}{SS}{#1}} 
\newcommand{\plt}[1]{\authornote{magenta}{TPLT}{#1}} %For explanation of the template
\newcommand{\an}[1]{\authornote{cyan}{Author}{#1}}

%% Common Parts

\newcommand{\progname}{ProgName} % PUT YOUR PROGRAM NAME HERE
\newcommand{\authname}{Team \#, Team Name
\\ Student 1 name
\\ Student 2 name
\\ Student 3 name
\\ Student 4 name} % AUTHOR NAMES                  

\usepackage{hyperref}
    \hypersetup{colorlinks=true, linkcolor=blue, citecolor=blue, filecolor=blue,
                urlcolor=blue, unicode=false}
    \urlstyle{same}
                                


\begin{document}

\maketitle
\thispagestyle{empty}

~\newpage

\pagenumbering{roman}

\begin{table}[hp]
\caption{Revision History} \label{TblRevisionHistory}
\begin{tabularx}{\textwidth}{llX}
\toprule
\textbf{Date} & \textbf{Developer(s)} & \textbf{Change}\\
\midrule
Oct 11 & All & Initial Draft\\
Jan 13 & Marlon Liu & Change based on reviews\\
... & ... & ...\\
\bottomrule
\end{tabularx}
\end{table}

~\newpage

\tableofcontents
\listoftables

~\newpage

\pagenumbering{arabic}


\section{Introduction}

This document provides the hazard analysis of the Hairesthetics application. Hairesthetics is an application that simulates 3D hairstyles. The definition of a hazard used throughout this document is based on Nancy Leveson's work. A hazard is any property or condition of Hairesthetics that has the potential to result in a loss in the system when paired with a condition in the environment. In Hairesthetics, there are hazards in safety (storing data), and security (restricting access to data). 

\section{Scope and Purpose of Hazard Analysis}
The scope of this document is to state any critical assumptions about the project, as well as to identify possible hazards within the system components, the effects and causes of failures, mitigation steps, and resulting safety and security requirements.

\section{System Boundaries and Components}
The system referred to in this document that the hazard analysis will be conducted on consists of:
\begin{enumerate}
    \item The iOS application installed on user devices, including both the front-end and back-end made up of the following four major components:
    \begin{itemize}
        \item Facial Recognition System
        \item Hair Modification System
        \item Hair Salon Recommendation System
        \item Authentication System       
    \end{itemize}
    \item User Device, iPhone
    \item \sout{MongoDB database where all data will be stored}
    \item \sout{Cloud database backup programs}
\end{enumerate}

\noindent The system boundary in this case includes the entire application, user device, \sout{the database, and the data backup program.} The user device and cloud hosting, hardware, \sout{and down-time of the database} are elements that are not controlled by this project. The user device is controlled by the user \sout{and portions of the database are controlled by MongoDB}. However, they are still critical elements of the system and were thus included in this hazard analysis.

\section{Critical Assumptions}
The application will be tested on Apple-provided iPhone simulators as well as developers' devices (iOS 14 and above). It is assumed that all physical devices are in good condition to be used for the project and if the application is compatible with Apple-provided iPhone simulators, then it is compatible and functions properly on actual iPhones with iOS 14 and above.\\
\\
\noindent The scope and specifications of the project will not change when the project takes place. However, when conducting the project, there might be cases where the scope and specifications need to be altered to cater to clients' requirements and needs of the project.\\
\\
\noindent \sout{The data will be automatically updated in the database, and all information in the database is synchronized.}


\section{Failure Mode and Effect Analysis}
\subsection{Hazards out of scope}
The out of scope hazards falls into the following categories:
\begin{itemize}
    \item User's Camera
    \item User's mobile phone hardware
    \item ArKit library
\end{itemize}
User side's hardware conditions is something that the team does not have control over. Besides, the ArKit library is also out of the scope since it is a third party library. The hazards can not be handled completely through our development, however we will try our best to minimize them.
\subsection{Failure Mode and Effect Analysis Table}
The failure modes \& effects analysis (FMEA) was chosen as a tool to identify and analyze the hazards within the system so that requirements can be made to mitigate them.

\wss{Include your FMEA table here}

\begin{landscape}
\begin{table}[h!]
\def\arraystretch{1.7}
\begin{tabularx}{1.2\textwidth} { |X|X|X|X|X
  | p{0.9cm}
  | p{0.9cm} | }
\hline
	\centering{Component} & \centering{Failure Mode} & \centering{Effects of Failure} & \centering{Causes of Failure} & \centering{Recommended Action} & \centering{SR} & Ref \\ \hline
    Facial Recognition System & The facial landmarks aren't computed properly & Inaccurate simulation results & User might not input proper image / video & Ask the user for inputs again & \sout{FR5, FR6} \textcolor{red}{IR6} & H1-1 \\ \cline{2-7}
     & System Crash & Overall application crash & Model execution failure & Display error message to prevent further actions & \sout{RAR1} \textcolor{red}{IR7} & H1-2 \\ \hline
     
     Hair Modification System & Failed to load hairstyle models & Virtual hair simulation won't appear on the screen  & \sout{Database failure} \textcolor{red}{File not exist, incorrect file format, file broken}  & \sout{Backup data or reboot} \textcolor{red}{manually check the file validity and replace if it's in bad condition} & \sout{HM6} \textcolor{red}{IR7} & H2-1 \\  \cline{2-7}
     
     ~ & Algorithm failed to detect hair coordinates  & User's hair color fails to change as selected  &  
     Hair coordination data failed to be retrieved. \newline 
     Facial recognition system crash
     & Refer to H1-1 & \sout{HM3 \newline HM4} \textcolor{red}{IR7} & H2-2 \\ \hline
     
     Hair Salon Recommendation System & Failed to compute recommendations, Invalid user input & User gets no recommendation and lose interests & \sout{Database failure} \textcolor{red}{Backend API fails to retrieve information, invalid GCP token} & \sout{Inform user to try again} \textcolor{red}{Update token in config and rerun the application} & \sout{HR2} \textcolor{red}{IR1, IR2} & H3-1 \\ \cline{2-7}
     ~ & System freeze or crash & system crash or overall application crash & \sout{Database failure}, Invalid user input & Backup data or reboot & \sout{HR1, SLR1} \textcolor{red}{IR7} & H3-2 \\ \hline
     

\end{tabularx}
\caption{Failure Mode and Effect Analysis Table}
\label{FMEA Table}
\end{table}
\end{landscape}


\newpage

\begin{landscape}
\begin{table}[h!]
\def\arraystretch{1.7}
\begin{tabularx}{1.2\textwidth} { |X|X|X|X|X
  | p{0.9cm}
  | p{0.9cm} | }
\hline
	\centering{Component} & \centering{Failure Mode} & \centering{Effects of Failure} & \centering{Causes of Failure} & \centering{Recommended Action} & \centering{SR} & Ref \\ \hline
     

     \sout{Authentication System} & \sout{User can not login to the app} & \sout{User can not use the functionality of the application}  & \sout{Login credentials do not match the records in the database} & \sout{Inform user to try again or reset credentials} & \sout{AR1} & \sout{H4-1} \\ \cline{2-7}
     ~ & \sout{System failed to use the default camera with the device}  & \sout{The application can not be used since there's no input source} & 
    \sout{User denied the access of the camera.}  \newline
    \sout{The device has no proper camera system.}
     & 
     \sout{Prompt the user to allow camera access.} \newline
     \sout{Try another device with a valid camera.}
     & \sout{AR2} \newline \sout{AR3} & \sout{H4-2} \\ \hline   

    General & App crashes unexpectedly & User loses the current progress  &
    The device runs out of battery. \newline 
    The application crashes due to instability.  &
    Charge the device. \newline 
    Reboot and restart the app. & 
    \sout{RAR1 \newline
    RAR2} \textcolor{red}{IR7}
    &  H5-1 \\  \cline{2-7}
    ~ & Failed to save the images locally & The user won't have the simulation result & User denied the system access to the local storage. \textcolor{red}{The local storage is full.} & Prompt the user to allow access to local storage, \textcolor{red}{Prompt users to clear up local storage before retrying} & \sout{AR2} \textcolor{red}{IR1, IR2, IR7} & H5-2 \\   
    \hline

\end{tabularx}
\caption{Failure Mode and Effect Analysis Table}
\label{FMEA Table}
\end{table}
\end{landscape}

\section{Safety and Security Requirements}

Requirements that have been included in Revision 0 of the Software Requirements Specification document are written in \textbf{bold}. 

\subsubsection{Access Requirements}
\begin{itemize}
    \item[ACR\refstepcounter{acrreqnum}\theacrreqnum \label{R_Inputs}:] \textbf{Users will be able to access images they previously stored.}
    \item[ACR\refstepcounter{acrreqnum}\theacrreqnum \label{R_Inputs}:]
    \textbf{Admins will be able to unlock locked resources.}
    \item[ACR\refstepcounter{acrreqnum}\theacrreqnum \label{R_Inputs}:]
    \textbf{Only admins will be able to modify application information.}
    \item[ACR\refstepcounter{acrreqnum}\theacrreqnum\label{R_Inputs}:] 
    Users will be able to access the history of the hairs they chose.
\end{itemize}


\subsubsection{Integrity Requirements}
\begin{itemize}
    \item[IR\refstepcounter{irreqnum}\theirreqnum \label{R_Inputs}:] \textbf{The product will not modify data unnecessary.}
    \textcolor{red}{The product will not touch or modify data/objects in the local storage.}
    \item[IR\refstepcounter{irreqnum}\theirreqnum \label{R_Inputs}:] \textbf{The product will not modify any data unrelated to its execution.}
    \textcolor{red}{The product will only access local storage for saving images, will not modify other data.}
    \item[IR\refstepcounter{irreqnum}\theirreqnum \label{R_Inputs}:] \sout{\textbf{Data will be automatically backed up daily.}}
    \item[IR\refstepcounter{irreqnum}\theirreqnum \label{R_Inputs}:] \sout{\textbf{Unsaved data will be stored locally on the user's device if it cannot be uploaded to the remote database and is not explicitly discarded.}}
    \textcolor{red}{The data will only be saved in local storage upon users' requests.}
    \item[IR\refstepcounter{irreqnum}\theirreqnum \label{R_Inputs}:] All locked resources will be unlocked once the user sends the request to admins and admins approves the request.
    \textcolor{red}{
    \item[IR\refstepcounter{irreqnum}\theirreqnum
    \label{R_Inputs}:] The data input is accurate, complete, and consistent.
    \item[IR\refstepcounter{irreqnum}\theirreqnum
    \label{R_Inputs}:] The application will execute fully functionally}
    
\end{itemize}

\subsubsection{Privacy Requirements}
\begin{itemize}
    \item[PRR\refstepcounter{prrreqnum}\theprrreqnum \label{R_Inputs}:] \textbf{Users will not be able to access data generated by other users.}
    \item[PRR\refstepcounter{prrreqnum}\theprrreqnum \label{R_Inputs}:] \textbf{Users will be required to register and login to the application with their emails.}
    \item[PRR\refstepcounter{prrreqnum}\theprrreqnum \label{R_Inputs}:] Admins will not able to access data generated by other users.
\end{itemize}

\subsubsection{Audit Requirements}
\begin{itemize}
    \item[AUR\refstepcounter{aurreqnum}\theaurreqnum \label{R_Inputs}:] \textbf{Requirements should be easy to read and verify against the system facilitate regular inspections.}
\end{itemize}

\subsubsection{Immunity Requirements}
N/A

\section{Roadmap}

The hazard analysis resulted in new safety and security requirements given in section above. A number of the requirements will be implemented within the capstone timeline such as ACR1, ACR3, IR1, IR2, IR4, PRR1, PRR2, PRR3, AUR1. However, some of them will not, due to project time constraints, such as ACR2, ACR4, IR3, IR5. These will be implemented in the future. Towards the end of the project, the hazard analysis will be consulted to gain an understanding of which risks have been successfully mitigated and which one sill require work.

\end{document}